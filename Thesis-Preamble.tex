%-----------------------------------------------
% Dateiname: Thesis-Preamble.tex
% Autor    : Stefano Kowalke <blueduck@gmx.net>
% Lizenz   : BSD
%-----------------------------------------------

%----------------------------------
% Dokumentenklasse DINA4 einseitig
%----------------------------------
\documentclass[
	fontsize=11pt,            % Die Schriftgröße
	twoside=false,            % scrbook hat per Default ein Zwei-Seitenlayout
	parskip=full,             % Steuert die Absätze. http://www.rrzn.uni-hannover.de/fileadmin/kurse/material/latex/scrguide.pdf Tabelle 3.7
	headsepline,              % Fügt eine Trennungslinie in den Seitenkopf
	footnotes=multiple,       % Fügt ein Komma zwischen den Indexzahlen bei aufeinanderfolgende Fußnoten ein
	numbers=noendperiod       % Keinen Punkt der letzten Gliederungsebene in der Überschrift  -> 1.2.1 statt 1.2.1.
]{scrbook}

%=================================================
% Angaben zur Arbeit wie Titel und Name des Autor
%=================================================
\makeatletter
    \title{Titel der Bachelorthesis}\let\Title\@title
    \author{Kathrin Janeway}        \let\Author\@author
    \date{\today}                   \let\Date\@date
\makeatother
\def\NameOfDepartment{Transportwesen}
\def\AuthorEmail{<kathrin.janeway@starfleet.aca.edu>}
\def\Major{Stellarkartographie}
\def\MatricleNumber{123456}

%===============
% Pakete laden
%===============
\usepackage{ngerman}                       % Umschalten auf die neue deutsche Rechtschreibung
\usepackage{fontspec}                      % Wird von LuLaTeX benöigt und löst "fontenc" ab.
\usepackage{polyglossia}                   % Wird von LuLaTeX benöigt und löst "babel" ab.
\usepackage{blindtext}                
\usepackage{hyperref}                      % Stellt Links in Schreibmaschinenschrift dar und legt einen Link über den Text. 
                                           % Dieses Package sollte als letztes aufgerufen werden, da es Problem mit Anderen geben könnte

%----------------
% Renew commands
%----------------
\renewcommand*{\multfootsep}{,\nobreakspace}  % Fügt bei den hochgestellten Indexzahlen von Fußnoten ein Leerzeichen nach dem Komma ein
\deffootnote{1em}{1em}{\thefootnotemark\ }    % Setzt die Indexzahlen in den Fußnoten etwas entfernt vom Text

%====================
% Paketeinstellungen
%====================

%-------------------
% Linkkonfiguration
%-------------------
\hypersetup{
	pdftitle={\Title},
	pdfauthor={\Author},
	pdfsubject={\Title},
	pdfcreator={\Author},
	pdfkeywords={typo3} {dbal} {doctrine} {mysql} {postgres},
	linktoc=all,
	colorlinks=true,
	linkcolor=black,
	citecolor=black,
	filecolor=black,
	urlcolor=blue,
}