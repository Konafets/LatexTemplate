%-----------------------------------------------
% Dateiname: Thesis-Preamble.tex
% Autor    : Stefano Kowalke <blueduck@gmx.net>
% Lizenz   : BSD
%-----------------------------------------------

%----------------------------------
% Dokumentenklasse DINA4 einseitig
%----------------------------------
\documentclass[
	fontsize=11pt,           % Die Schriftgröße
	twoside=false,           % scrbook hat per Default ein Zwei-Seitenlayout
	parskip=full,             % Steuert die Absätze. http://www.rrzn.uni-hannover.de/fileadmin/kurse/material/latex/scrguide.pdf Tabelle 3.7
	headsepline,              % Fügt eine Trennungslinie in den Seitenkopf
	footnotes=multiple
]{scrbook}

\usepackage{ngerman}                       % Umschalten auf die neue deutsche Rechtschreibung
\usepackage{fontspec}                      % Wird von LuLaTeX benöigt und löst "fontenc" ab.
\usepackage{polyglossia}                   % Wird von LuLaTeX benöigt und löst "babel" ab.
\usepackage{blindtext}

%----------------
% Renew commands
%----------------
\renewcommand*{\multfootsep}{,\nobreakspace}  % Fügt bei den hochgestellten Indexzahlen von Fußnoten ein Leerzeichen nach dem Komma ein
\deffootnote{1em}{1em}{\thefootnotemark\ }    % Setzt die Indexzahlen in den Fußnoten etwas entfernt vom Text
